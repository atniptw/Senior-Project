
\documentclass{article}
\usepackage{fullpage}
\usepackage{color}
\usepackage[normalem]{ulem}
\usepackage{hyperref}
\usepackage{enumitem}
\hypersetup{colorlinks}
\hyphenpenalty=100000
\begin{document}
\setlength{\voffset}{3.5in}
\title{Metrics}
\author{\Large Android Based Situational Awareness: Moving Map\\
Tom Atnip, Susi Cisneros, Sam Kim, and Seth Troisi}
\date{\today}
\maketitle
\clearpage
\setlength{\voffset}{0pt}
\tableofcontents
\clearpage
~\\
\begin{Large}\textbf{Changes (based off Git commits)}\end{Large}\\
~\\
\begin{tabular}{ | p{1.5in} | p{4.5in} | }
\hline
\textbf{Date} & \textbf{Description}\\
\hline
\hline
September 13, 2012 & Document started\\
\hline
September 20, 2012 & Created outline\\
\hline
September 23, 2012 & Wrote Documentation Metrics and Code Metrics\\
\hline
September 24, 2012 & Updated Requirements as per Gate 5 visit with Raytheon\\
\hline
October 4, 2012 & Separate document created\\
\hline
October 8, 2012 & Revised previous metrics and added all other metrics\\
\hline
\end{tabular}
\clearpage

\section{Introduction}

\section{Metrics}
\subsection{Project}
\subsubsection{Documentation}
The progress of the documentation will be tracked by breaking it down into three parts: the percent written and ready for review, the percent that has been reviewed, and the percent that is ready for delivery. The initial portion will encompass the percentage of the requirements, features, and other material that have been documented according to our currently known goals. A portion of the documentation will be considered in the reviewed stage once Dr. Wollowski and/or JD Hill have provided feedback and approval. Once a section of the documentation is in its final state (written, reviewed, and stable), it will be considered complete.\\ \\
\begin{tabular}{l r}
Percent Written: & 0 \\
Percent Reviewed: & 0 \\
Percent Complete: & 0 \\
\end{tabular}
\subsubsection{Code}
During the coding phase of this project, progress will be tracked by the features scheduled during an iteration and the number of features completed. Code will belong to one of five phases – unwritten, written, peer reviewed, tested, or complete. Once code has been written and passes the required unit tests, it will undergo a peer review to check for good coding practices, clarity, and errors. After a peer review the functionality will then be required to pass integration tests. Once it has passed system integration, it will be considered complete and will be merged into the main branch of code.\\ \\
\begin{tabular}{l r}
Percent Written: & 0 \\
Percent Reviewed: & 0 \\
Percent Tested: & 0 \\
Percent Complete: & 0 \\
\end{tabular}
\subsubsection{Testing}
For the final phase of the project, progress will be measured by how many tests are passing. The tests that the software will be subjected to will be more thorough than the tests required for code to join the main branch. Most tests will be automated, but there will also be human factor tests.\\ \\
\begin{tabular}{l r}
Percent Passing: & 0 \\
\end{tabular}
\subsection{Process}
The process that the project is following will be measured by due dates met versus missed due dates.\\ \\
\begin{tabular}{l r}
Milestone Dates Kept: & 0 \\
\end{tabular}

\subsection{Communication}
Communication will be measured by how well the team feels that their needs are being heard and being taken care of, along with efficiency of meetings.\\ \\
\begin{tabular}{l r}
Team Confidence: & 0 \\
Meetings: & 0 \\
\end{tabular}

\end{document}