
\documentclass{article}
\usepackage{fullpage}
\usepackage{color}
\usepackage[normalem]{ulem}
\usepackage{hyperref}
\usepackage{enumitem}
\hypersetup{colorlinks}
\hyphenpenalty=100000
\begin{document}
\setlength{\voffset}{3.5in}
\title{Milestone 1}
\author{\Large Android Based Situational Awareness: Moving Map\\
Tom Atnip, Susi Cisneros, Sam Kim, and Seth Troisi}
\date{14 September 2012}
\maketitle
\clearpage
\setlength{\voffset}{0pt}
\tableofcontents
\clearpage
~\\
\begin{Large}\textbf{Changes (based off Git commits)}\end{Large}\\
~\\
\begin{tabular}{ | p{1.5in} | p{4.5in} | }
\hline
\textbf{Date} & \textbf{Description}\\
\hline
\hline
13 September 2012 & Document started\\
\hline
13 September 2012 & Initial work on document\\
\hline
\end{tabular}
\clearpage

\section{Introduction}

\section{Problem Statement}
\subsection{User/Stakeholder Descriptions}
\subsubsection{Users}
\textbf{Soldier, Police Officers, and other Ground Personnel}
The users of our program are seeking to maintain their situational awareness in locations which may not have connectivity to the Internet.  Many of them use voice guided situational awareness technology, but in light of advances in mobile devices, they could receive this information in a visual manner.  

subsubsection{Stakeholders}
\textbf{Raytheon}
Raytheon's customers are military organizations, many of which are using Raytheon's current situational awareness technologies.  Raytheon is looking to update these technologies to keep their position as a leading provider of military systems.

\textbf{JD Hill}
JD is the client who proposed this solution.  He is a major proponent of using mobile devices in a military application.
\textbf{Doug Duesseau}
Doug is the acting Technical Lead for this project. 
\textbf{Development Team}
The development team on this project are graduating seniors who wish to learn more about the software development process and interaction with a client.  They are very interested in learning more about developing on the Android platform.

\subsection{Key Needs}
\subsection{Current Solution}
\subsection{Alternatives}


\section{Functional Requirements}

\subsection{General}

\begin{enumerate}[label*=3.1.\arabic*]
\item Android Moving Map shall provide the ability to pan the map by a dragging gesture
\item Android Moving Map shall provide the ability to zoom using double tap, pinch gestures, or using an on-screen button
\item Android Moving Map shall display map tiles that are either stored on the device or provided by a local server
\item Android Moving Map shall display other relevant information supplied by a local server
\item Android Moving Map shall georeference the location of the device
\end{enumerate}

\section{Non-functional Requirements}

\subsection{General}

\begin{enumerate}[label*=4.1.\arabic*]
\item Android Moving Map shall run on Android platforms running at least version 3.0 (Honeycomb)
\item Android Moving Map shall be able to receive GPS data from a local server or the device
\item Android Moving Map shall display properly on either mobile phones or tablets
\end{enumerate}

\section{Project Plan}
\subsection{Schedule}
\subsection{Risks}

\section{Metrics}
\subsection{Project}
\subsection{Process}
\subsection{Communication}

\clearpage

\section{Questions}

\begin{enumerate}[label*=3.\arabic*]
\item Should the orientation be north up or heading up or should we include both? If so what should the default be?
\item What is the "other relevant information" that will be displayed on the map?
\item Will we have to display multiple types of map tiles (i.e. satellite or street)?
\item Will we have any control/knowledge in how map tile data is sent to the device (i.e. filetype and format)?
\item Can we get sample map data from Raytheon?
\item What will be the restrictions on zoom level, since it alters how many tiles need to be stored on the device?
\item When/how should the device pull map tile data from the server?
\item Does the view follow the user as he/she travels?
\item Does the server have connection to the Internet?
\item What ways should the device connect to the server?
\item Will there be any views aside from the map?
\item What happens if the connection is lost?
\item What happens when connection is regained?
\item How is the connection initially established?
\item Should we disable the device from turning off the display?
\item What happens during/after a critical error with the device?
\item Can this be an open source project, as we may run into GPL licensing issues?
\item Who is maintaining code after the project finishes?
\item Who is going to be using the system?
\item What existing functionality needs to be carried over?
\end{enumerate}


\end{document}