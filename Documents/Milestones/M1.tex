
\documentclass{article}
\usepackage{fullpage}
\usepackage{color}
\usepackage[normalem]{ulem}
\usepackage{hyperref}
\usepackage{enumitem}
\hypersetup{colorlinks}
\hyphenpenalty=100000
\begin{document}
\setlength{\voffset}{3.5in}
\title{Milestone 1}
\author{\Large Android Based Situational Awareness: Moving Map\\
Tom Atnip, Susi Cisneros, Sam Kim, and Seth Troisi}
\date{14 September 2012}
\maketitle
\clearpage
\setlength{\voffset}{0pt}
\tableofcontents
\clearpage
~\\
\begin{Large}\textbf{Changes (based off Git commits)}\end{Large}\\
~\\
\begin{tabular}{ | p{1.5in} | p{4.5in} | }
\hline
\textbf{Date} & \textbf{Description}\\
\hline
\hline
13 September 2012 & Document started\\
\hline
13 September 2012 & Wrote Requirements and Questions to ask Raytheon\\
\hline
20 September 2012 & Created outline\\
\hline
20 September 2012 & Wrote Users/Stakeholders section\\
\hline
23 September 2012 & Wrote Key Needs, Alternatives, Risks, Documentation Metrics, and Code Metrics\\
\hline
24 September 2012 & Updated Requirements as per Gate 5 visit with Raytheon\\
\hline
\end{tabular}
\clearpage

\section{Introduction}

\section{Problem Statement}
\subsection{User/Stakeholder Descriptions}
\subsubsection{Users} 
\textbf{Soldier, Police Officers, and other Ground Personnel}\\
The users of our program are seeking to maintain their situational awareness in locations which may not have connectivity to the Internet.  Many of them use voice guided situational awareness technology, but in light of advances in mobile devices, they could receive this information in a visual manner.  

\subsubsection{Stakeholders}
\textbf{Raytheon}\\
Raytheon's customers are mainly military organizations, many of which are using Raytheon's current situational awareness technologies.  Raytheon is looking to update these technologies to keep their position as a leading provider of military systems.\\ \\
\textbf{JD Hill}\\
JD is the client who proposed this solution.  He is a major proponent of using mobile devices in a military application.\\ \\
\textbf{Doug Duesseau}\\
Doug is the acting Technical Lead for this project. \\ \\
\textbf{Development Team}\\
The development team on this project are graduating seniors who wish to learn more about the software development process and interaction with a client.  They are very interested in learning more about developing on the Android platform.

\subsection{Key Needs}

\begin{tabular}{ | p{.5in} | p{4.5in} | }
\hline
\textbf{ID} & \textbf{Need}\\
\hline
\hline
N0 & View map of surrounding area\\
\hline
N1 & View points of interest on the map\\
\hline
N2 & View current location on the map\\
\hline
N3 & Map must not require internet access\\
\hline
N4 & Map must be Android based\\
\hline
N5 & Application must work on any size android device\\
\hline
\end{tabular}


\subsection{Current Solution}

\subsection{Alternatives}
All considered solutions to the proposed system require Internet access.

\section{Requirements}

\subsection{Functional}

\begin{tabular}{ | p{.5in} | p{5in} | }
\hline
\textbf{ID} & \textbf{Requirement}\\
\hline
\hline
FR0 & Ability to pan the map by a dragging gesture\\
\hline
FR1 & Ability to zoom using double tap, pinch gestures, or using an on-screen button\\
\hline
FR2 & Display map tiles that are either stored on the device or provided by a local server\\
\hline
FR3 & Display other relevant information stored in a layer\\
\hline
FR4 & Georeference the location of the device\\
\hline
FR5 & Center on current location by pressing a button\\
\hline
FR6 & Choose map type by selecting from a list\\
\hline
FR7 & Choose the layers shown by selecting from a list\\
\hline
FR8 & Change default settings via a settings menu found in the menu bar\\
\hline
FR9 & Display a compass\\
\hline
FR10 & Toggle heading/north up by clicking the compass\\
\hline
FR11 & Access a help menu via the menu bar\\
\hline
FR12 & Add custom points of interest\\
\hline
\end{tabular}

\subsection{Non-functional}

\begin{tabular}{ | p{.5in} | p{5in} | }
\hline
\textbf{ID} & \textbf{Requirement}\\
\hline
\hline
NR0 & Run on Android platforms running at least version 3.0 (Honeycomb)\\
\hline
NR1 & Ability to receive GPS data from a local server or the device\\
\hline
NR2 & Display properly on either mobile phones or tablets\\
\hline
NR3 & Modular code\\
\hline
\end{tabular}

\section{Project Plan}
\subsection{Schedule}
\subsection{Risks}

\begin{tabular}{ | p{.5in} | p{4.5in} | }
\hline
\textbf{ID} & \textbf{Risk}\\
\hline
\hline
R0 & Performance of the system\\
\hline
R1 & Finding a feature complete mapping engine\\
\hline
R2 & Organizing data in the correct format in a timely manner\\
\hline
\end{tabular}

\section{Metrics}
\subsection{Project}
\subsubsection{Documentation}
This will encompass the percentage of the code, features, and other material that has been documented\\
Percent Written:	\\
This will encompass the percentage of the documentation that has been reviewed by Dr. Wollowski and/or JD Hill\\
Percent Reviewed:	0\\
This will encompass the percentage of the documentation is in a final state (written, reviewed, and stable).\\
Percent Complete:	0\\

\subsubsection{Code}
This will encompass the percentage of the code, by feature, that has been written\\
Percent Written:	0\\
This will encompass the percentage of the code that has passed Code review (both Interal and External)\\
Percent Reviewed:	0\\
This will encompass the percentage of the code that has passed testing phase.\\
Percent Tested:		0\\
This will encompass the percentage of the code that is in a final reviewed, tested, and stable form.\\
Percent Complete:	0\\


\subsubsection{Testing}
\subsection{Process}
\subsection{Communication}

\clearpage

\section{Questions}

\begin{enumerate}[label*=3.\arabic*]
\item Should the orientation be north up or heading up or should we include both? If so what should the default be?
\item What is the "other relevant information" that will be displayed on the map?
\item Will we have to display multiple types of map tiles (i.e. satellite or street)?
\item Will we have any control/knowledge in how map tile data is sent to the device (i.e. filetype and format)?
\item Can we get sample map data from Raytheon?
\item What will be the restrictions on zoom level, since it alters how many tiles need to be stored on the device?
\item When/how should the device pull map tile data from the server?
\item Does the view follow the user as he/she travels?
\item Does the server have connection to the Internet?
\item What ways should the device connect to the server?
\item Will there be any views aside from the map?
\item What happens if the connection is lost?
\item What happens when connection is regained?
\item How is the connection initially established?
\item Should we disable the device from turning off the display?
\item What happens during/after a critical error with the device?
\item Can this be an open source project, as we may run into GPL licensing issues?
\item Who is maintaining code after the project finishes?
\item Who is going to be using the system?
\item What existing functionality needs to be carried over?
\end{enumerate}


\end{document}