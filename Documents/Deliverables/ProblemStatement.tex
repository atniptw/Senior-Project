
\documentclass{article}
\usepackage{fullpage}
\usepackage{color}
\usepackage[normalem]{ulem}
\usepackage{hyperref}
\usepackage{enumitem}
\hypersetup{colorlinks}
\hyphenpenalty=100000
\begin{document}
\setlength{\voffset}{3.5in}
\title{Problem Statement}
\author{\Large Android Based Situational Awareness: Moving Map\\
Tom Atnip, Susi Cisneros, Sam Kim, and Seth Troisi}
\date{\today}
\maketitle
\clearpage
\setlength{\voffset}{0pt}
\tableofcontents
\clearpage
~\\
\begin{Large}\textbf{Changes}\end{Large}\\
~\\
\begin{tabular}{ | p{1.5in} | p{4.5in} | }
\hline
\textbf{Date} & \textbf{Description}\\
\hline
\hline
September 13, 2012 & Document started\\
\hline
September 20, 2012 & Created outline\\
\hline
September 20, 2012 & Wrote Users/Stakeholders section\\
\hline
September 23, 2012 & Wrote Key Needs, Alternatives\\
\hline
October 4, 2012 & Separate document created\\
\hline
October 15, 2012 & Added Introduction\\
\hline
January 26, 2013 & Project Document Audit\\
\hline
February 5, 2013 & Current Solution update\\
\hline
\end{tabular}
\clearpage

\section{Introduction}
This document will identify the users, stakeholder, and key needs for the
project. It will also identify key features and quality attributes.

\section{Problem Statement}
\subsection{User/Stakeholder Descriptions}
\subsubsection{Users} 
\textbf{Soldier, Police Officers, and other Ground Personnel}\\
The users of our program are seeking to maintain their situational awareness in locations which may not have connectivity to the Internet.  Many of them use voice guided situational awareness technology, but in light of advances in mobile devices, they could receive this information in a visual manner.  

\subsubsection{Stakeholders}
\textbf{Raytheon}\\
Raytheon's customers are mainly military organizations, many of which are using Raytheon's current situational awareness technologies.  Raytheon is looking to update these technologies to keep their position as a leading provider of military systems.\\ \\
\textbf{JD Hill}\\
JD is the client who proposed this solution.  He is a major proponent of using mobile devices in a military application.\\ \\
\textbf{Doug Duesseau}\\
Doug is the acting Technical Lead for this project. \\ \\
\textbf{Development Team}\\
The development team on this project are graduating seniors who wish to learn more about the software development process and interaction with a client.  They are very interested in learning more about developing on the Android platform.

\subsection{Key Needs}

\begin{tabular}{ | p{.5in} | p{4.5in} | }
\hline
\textbf{ID} & \textbf{Need}\\
\hline
\hline
N0 & View map of surrounding area\\
\hline
N1 & View points of interest on the map\\
\hline
N2 & View current location on the map\\
\hline
N3 & Map must not require internet access\\
\hline
N4 & Map must be Android based\\
\hline
N5 & Application must work on any size android device\\
\hline
\end{tabular}


\subsection{Current Solution}
The chosen mapping engine was OSMDroid. This engine came pre-built with offline map support, which handled the key functional requirement. The engine is also part of the Open source community which allows the code to be used free of charge, as long as licensing requests from the owner are met. There is also built in support for overlays.

\subsection{Alternatives}
All considered solutions to the proposed system require Internet access. Most other mapping engines explored were very domain specific (ie hiking or biking) and lacked significant amounts of documentation.

\end{document}
